\documentclass[12pt,letterpaper]{hmcpset}
\usepackage[margin=1in]{geometry}
\usepackage{graphicx}
\usepackage{amsmath,amssymb}

% info for header block in upper right hand corner
\name{Evan Hubinger}
\class{Physics 24a - Section 1}
\assignment{Work and Energy}
\duedate{Monday, February 29, 2016}

\renewcommand{\labelenumi}{{(\alph{enumi})}}
\newcommand{\diagram}[1]{\begin{center} \includegraphics[width=2in]{#1} \end{center}}

\begin{document}

\problemlist{5.\{2,5,7,8,10\}}

\begin{problem}[1 - Block, spring, and friction - KK 5.2]
    A block of mass $M$ slides along a
    horizontal table with speed $v_{0}$. At $x
    = 0$ it hits a spring with spring constant
    $k$ and begins to experience a friction
    force, as indicated in the sketch. The
    coefficient of friction is variable and
    is given by $\mu = b x$, where $b$ is a
    constant. Find the distance $l$ the block
    travels before coming to a permanent stop.
    \diagram{img/5_2.png}
\end{problem}

\begin{solution}
    \vfill
\end{solution}
\newpage

\begin{problem}[2 - Work on a whirling mass - KK 5.5]
    Mass $m$ whirls on a frictionless table,
    held to circular motion by a string
    which passes through a hole in the table.
    The string is pulled so that the
    radius of the circle changes from $r_i$ to $r_f$.
    \begin{enumerate}
    \item Show that the quantity $L = m r^2 \dot{\theta}$
        remains constant.
    \item Show that the work in pulling the
        string equals the increase in
        kinetic energy of the mass.
    \end{enumerate}
\end{problem}

\begin{solution}
    \vfill
\end{solution}
\newpage

\begin{problem}[3 - Beads on hanging ring* - KK 5.7]
  A ring of mass $M$ hangs from a thread, and two beads of mass $m$ slide on it
  without friction, as shown. The beads are released simultaneously from the top
  of the ring and slide down opposite sides. Show that the ring will start to
  rise if $m > 3M/2$, and find the angle at which this occurs.
  \diagram{img/5_7.png}
\end{problem}

\begin{solution}
    \vfill
\end{solution}
\newpage

\begin{problem}[4 - Damped oscillation* - KK 5.8]
  The block shown in the drawing is acted on by a spring with spring constant
  $k$ and a weak friction force of constant magnitude $f$. The block is pulled
  distance $x_{0}$ from equilibrium and released. It oscillates many times and
  eventually comes to rest.
  \begin{enumerate}
    \item Show that the decrease of amplitude is the same for each cycle of
      oscillation.
    \item Find the number of cycles $n$ the mass oscillates before coming to
      rest.
  \end{enumerate}
  \diagram{img/5_8.png}
\end{problem}

\begin{solution}
    \vfill
\end{solution}
\newpage

\begin{problem}[5 - Falling chain* - KK 5.10]
  A chain of total mass $M$ and length $l$ is suspended vertically with its
  lowest end touching a scale. The chain is released and falls onto the
  scale. What is the reading of the scale when a length of chain, $x$, has
  fallen? (Neglect the size of individual links.)
  \diagram{img/5_10.png}
\end{problem}

\begin{solution}
    \vfill
\end{solution}
\newpage

\end{document}
